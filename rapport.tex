\documentclass{article}
\usepackage[french]{babel}
\usepackage[a4paper, total={6in, 8in}]{geometry}
\usepackage{hyperref}
\hypersetup{
    linktoc=all, %set to all if you want both sections and subsections linked
}
%\usepackage{cite}
%\usepackage{lipsum}

%\usepackage{authblk} %FOR CENTER AUTHOR IMPORTANT !!! DONT ERASE

\title{{\huge UE Ouverture Professionnelle\\« Découverte des métiers et des formations »}}
\author{$ $\\{\LARGE Sacha Duperret}\\ $ $\\ \href{mailto:sduperret@u-bordeaux.fr}{sduperret@u-bordeaux.fr}\\22001114,\\MINF201B4,\\Université de Bordeaux,\\Talence,\\France}
\date{15 mai 2022}


%Graphicx
\usepackage{graphicx} %Loading the package
\graphicspath{{images/}} %path to figures


\begin{document}
\maketitle

\vspace{20pt}

\tableofcontents

\vspace{50pt}

\newpage \section{Introduction}
Durant notre vie, notre conception du travail évolue. Tout comme nos objectifs professionnels peuvent changer, suivant nos attentes, connaissances et opportunités. Dans ce rapport, je vais essayer de mettre en évidence mes recherches sur les opportunités d'emploi dans un monde mêlant informatique et médecine. Je choisis ce thème car il correspond à mes attentes actuelles, pour mon avenir professionnel.\\

Avant ce travail, et pour ne pas commencer mes études dans l'inconnu, je m'étais fortement renseigné. Formations, fonctions/missions, salaire et modalités d'exercice m'étaient plutôt familiers. Ces informations, facilement accessibles sur internet ou auprès de réseaux professionnels comme Linkedin, m'ont permis d'affiner mon choix et la formation initiale que j'ai choisi de suivre. Des ressources plus spécialisées sur youtube ou auprès de l'espace orientation carrière sont aussi disponible, avec un diagramme représentant les domaines de débouchés possibles qui donne une idée large mais précise des possibilités d'emploi après licence, master ou doctorat. Me manquaient des informations plus pratique, le stack technique (les technologies, logiciels, langages, ...) employés, les délais de réalisation et le degré de complexité des projets (bien que ceci soit très variable suivant le secteur). Au départ, et comme on peut parfois l'entendre, je me demandais quelle était l'insertion des jeunes diplômés à bac+5 dans des entreprises porteuses.Question vite et bien répondu par la réunion de rentrée de L1 mathématiques-informatiques, mais aussi les sollicitations des entreprises dès que informatique a été mentionné sur mon profil Linkedin.

\newpage \section{Recherche documentaire}
Durant mes recherches, je me suis concentré sur l'approfondissement de certains points : débouchés professionnels, catégorie/salaire d'emploi en sortie de formation, ``places disponibles'' en entreprise.\\

Pour cela, et pour avoir une vision d'ensemble, j'ai consulté le dossier thématique de septembre 2021 et plus particulièrement le poster ``Après des études en informatique : Éventail des débouchés''\footnote{\href{https://pro-fildoc.u-bordeaux.fr/index.php?lvl=notice_display&id=6890}{Lien cliquable document 1}}, mis en ligne par le SUIO de Nantes et consulté en mai 2022 sur l'espace orientation carrière de l'UB (Université de Bordeaux). A travers ce dernier, j'ai pu me familiariser avec les métiers qui sont ouverts aux informaticiens, classés selon les domaines : systèmes informatiques, réseaux et télécommunications, santé, études et développement, enseignement et recherche. Parmi ces derniers, une préférence pour la santé, l'étude et développement de solution ou l'enseignement domine. Cette recherche confirme mes attentes sur les domaines et la diversité des opportunités, tout comme la liberté d'évolution inter-domaines. J'apprends des débouchés spécifique en bio-informatique ou formateur. cela n'infirme rien de ce que j'avais découvert. La possibilité de devenir formateur m'intrigue.

Dans un second temps, j'ai consulté le site web du CEA (Commissariat à l'énergie atomique et aux énergies alternatives) dans un article de quelques pages intitulé \\Ingénieur-Chercheur en intelligence articielle''\footnote{\href{https://www.cea.fr/comprendre/jeunes/Pages/metiers/electronique-informatique-mathematiques/ingenieur-chercheur-en-intelligence-articielle.aspx}{Lien cliquable document 2}}. Publié le 31 août 2021, ce document explique de manière simplifiée les missions de l'ingénieur en intelligence artificielle. Il met aussi l'accent sur le quotidien du professionnel. Le large domaine d'application des connaissances que l'on nous enseigne en informatique est confirmé, avec ici un volte recherche qui est présent au delà de l'intitulé du poste mais aussi dans les missions quotidiennes. J'apprends la possibilité de faire de la recherche en informatique en dehors de laboratoire conventionnels, bien que je ne devrai pas l'être vu le large éventail que couvre les sciences de l'ordinateur. Mon choix, lui, n'évolue pas mais cette découverte m'ouvre des possibilités nouvelles en terme d'évolution. Volet formation, je reste convaincu que si un jour je veux exercer dans un milieu de recherche, des études universitaires sont nécessaires.

Ensuite, pour confirmer ce que j'ai pu observer dans l'aspect économique, j'étudie le site d'une école d'informatique Iris media school. Ce document de quelques paragraphes comprend un volet intitulé ``Plus que jamais, le secteur de l'informatique recrute''. Je peux ici confirmer par des chiffres (provenant d'études) mes impressions quand aux recrutement dans le monde de l'IT (InformaTique). Ainsi, ``en 2016, 31\%des recrutements concerneraient des jeunes diplômés de moins de 1 an d'expérience''. Cela affiche aussi, dans une moindre mesure, une forte migration inter-entreprise étant donné que la population observée n'a que peu d'expérience en entrant dans le monde professionnel et se fait embaucher à nouveau moins d'un an plus tard (à vérifier en comparant avec d'autres études). Cette étude m'indique donc la nécessité d'une certaine mobilité et adaptabilité sont nécessaire. Je considère cela comme un challenge plutôt qu'un ``mauvais'' point, même si ça pourrait le devenir après une dizaine d'année.

\newpage \section{Entretien avec des professionnels et connaissance de leur environnement}

\newpage \section{Discussion}

\newpage \section{Conclusion}

\newpage \section{Annexes}
%\subsection{Interview 1 de M. PIERRE HERICOURT}

%\subsection{Interview 1 de M. QUENTIN COASTALIOU}

%\subsection{Fiche état des lieux remplie}

%\subsection{Fiche recherche documentaire remplie}

\end{document}