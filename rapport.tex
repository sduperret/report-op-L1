\documentclass{article}
\usepackage[french]{babel}
\usepackage[a4paper, total={6in, 8in}]{geometry}
\usepackage{hyperref}
\hypersetup{
    linktoc=all, %set to all if you want both sections and subsections linked
}
%\usepackage{cite}
%\usepackage{lipsum}

%\usepackage{authblk} %FOR CENTER AUTHOR IMPORTANT !!! DONT ERASE

\title{{\huge UE Ouverture Professionnelle\\« Découverte des métiers et des formations »}}
\author{$ $\\{\LARGE Sacha Duperret}\\ $ $\\ \href{mailto:sduperret@u-bordeaux.fr}{sduperret@u-bordeaux.fr}\\22001114,\\MINF201B4,\\Université de Bordeaux,\\Talence,\\France}
\date{15 mai 2022}


%Graphicx
\usepackage{graphicx} %Loading the package
\graphicspath{{images/}} %path to figures


\begin{document}
\maketitle

\vspace{20pt}

\tableofcontents

\vspace{50pt}

\newpage \section{Introduction}
Durant notre vie, notre conception du travail évolue. Tout comme nos objectifs professionnels peuvent changer, suivant nos attentes, connaissances et opportunités. Dans ce rapport, je vais essayer de mettre en évidence mes recherches sur les opportunités d'emploi dans un monde mêlant informatique et médecine. Je choisis ce thème car il correspond à mes attentes actuelles, pour mon avenir professionnel.\\

Avant ce travail, et pour ne pas commencer mes études dans l'inconnu, je m'étais fortement renseigné. Formations, fonctions/missions, salaire et modalités d'exercice m'étaient plutôt familiers. Ces informations, facilement accessibles sur internet ou auprès de réseaux professionnels comme Linkedin, m'ont permis d'affiner mon choix et la formation initiale que j'ai choisi de suivre. Des ressources plus spécialisées sur youtube ou auprès de l'espace orientation carrière sont aussi disponible, avec un diagramme représentant les domaines de débouchés possibles qui donne une idée large mais précise des possibilités d'emploi après licence, master ou doctorat. Me manquaient des informations plus pratique, le stack technique (les technologies, logiciels, langages, ...) employés, les délais de réalisation et le degré de complexité des projets (bien que ceci soit très variable suivant le secteur). Au départ, et comme on peut parfois l'entendre, je me demandais quelle était l'insertion des jeunes diplômés à bac+5 dans des entreprises porteuses.Question vite et bien répondu par la réunion de rentrée de L1 mathématiques-informatiques, mais aussi les sollicitations des entreprises dès que informatique a été mentionné sur mon profil Linkedin.

\newpage \section{Recherche documentaire}
Durant mes recherches, je me suis concentré sur l'approfondissement de certains points : débouchés professionnels, catégorie/salaire d'emploi en sortie de formation, ``places disponibles'' en entreprise.\\

Pour cela, et pour avoir une vision d'ensemble, j'ai consulté le dossier thématique de septembre 2021 et plus particulièrement le poster ``Après des études en informatique : Éventail des débouchés''\footnote{\href{https://pro-fildoc.u-bordeaux.fr/index.php?lvl=notice_display&id=6890}{Lien cliquable document 1}}, mis en ligne par le SUIO de Nantes et consulté en mai 2022 sur l'espace orientation carrière de l'UB (Université de Bordeaux). A travers ce dernier, j'ai pu me familiariser avec les métiers qui sont ouverts aux informaticiens, classés selon les domaines : systèmes informatiques, réseaux et télécommunications, santé, études et développement, enseignement et recherche. Parmi ces derniers, une préférence pour la santé, l'étude et développement de solution ou l'enseignement domine. Cette recherche confirme mes attentes sur les domaines et la diversité des opportunités, tout comme la liberté d'évolution inter-domaines. J'apprends des débouchés spécifique en bio-informatique ou formateur. cela n'infirme rien de ce que j'avais découvert. La possibilité de devenir formateur m'intrigue.

Dans un second temps, j'ai consulté le site web du CEA (Commissariat à l'énergie atomique et aux énergies alternatives) dans un article de quelques pages intitulé \\Ingénieur-Chercheur en intelligence articielle''\footnote{\href{https://www.cea.fr/comprendre/jeunes/Pages/metiers/electronique-informatique-mathematiques/ingenieur-chercheur-en-intelligence-articielle.aspx}{Lien cliquable document 2}}. Publié le 31 août 2021, ce document explique de manière simplifiée les missions de l'ingénieur en intelligence artificielle. Il met aussi l'accent sur le quotidien du professionnel. Le large domaine d'application des connaissances que l'on nous enseigne en informatique est confirmé, avec ici un volte recherche qui est présent au delà de l'intitulé du poste mais aussi dans les missions quotidiennes. J'apprends la possibilité de faire de la recherche en informatique en dehors de laboratoire conventionnels, bien que je ne devrai pas l'être vu le large éventail que couvre les sciences de l'ordinateur. Mon choix, lui, n'évolue pas mais cette découverte m'ouvre des possibilités nouvelles en terme d'évolution. Volet formation, je reste convaincu que si un jour je veux exercer dans un milieu de recherche, des études universitaires sont nécessaires.

Ensuite, pour confirmer ce que j'ai pu observer dans l'aspect économique, j'étudie le site d'une école d'informatique Iris media school. Ce document de quelques paragraphes comprend un volet intitulé ``Plus que jamais, le secteur de l'informatique recrute''. Je peux ici confirmer par des chiffres (provenant d'études) mes impressions quand aux recrutement dans le monde de l'IT (InformaTique). Ainsi, ``en 2016, 31\%des recrutements concerneraient des jeunes diplômés de moins de 1 an d'expérience''. Cela affiche aussi, dans une moindre mesure, une forte migration inter-entreprise étant donné que la population observée n'a que peu d'expérience en entrant dans le monde professionnel et se fait embaucher à nouveau moins d'un an plus tard (à vérifier en comparant avec d'autres études). Cette étude m'indique donc la nécessité d'une certaine mobilité et adaptabilité sont nécessaire. Je considère cela comme un challenge plutôt qu'un ``mauvais'' point, même si ça pourrait le devenir après une dizaine d'année.

\newpage \section{Entretien avec des professionnels et connaissance de leur environnement}
\begin{center}
\begin{figure}[h!]
	\includegraphics[scale=0.52]{tab-interview}
\end{figure}
\end{center}
%------------------------------------------

J'ai choisi de décrire le métier d'ingénieur de formation en informatique, de façon générale.
Un ingénieur de formation en informatique est un professionnel du domaine de l'informatique, diplômé d'un bac+5. Son statut de poste est très variable, il peut travailler dans le secteur publique comme dans le privée, ou être indépendant. Le type de contrat peut aussi beaucoup varier : CDI, CDD, missions, temps plein ou partiel. Aux deux parties de choisir. La durée hebdomadaire de travail dépend du contrat mais peut aller jusqu'à 37h30 dans le cas d'un CDI dans le public. \\

Les tâches principale de l'informaticien sont, en fonction de sa spécialité, très vastes. Pour un diplômé en génie logiciel, ce serait conceptualiser des solutions informatiques à des problèmes informatiques ou non. Il y aurait aussi un travail de veille, du point de vue des technologies utilisées, améliorations à apporter, correctifs de sécurité, ... Les outils utilisés dépendent du domaine (public/privé), de la compagnie et du secteur (santé, cybersécurité, ...). Généralement, le plus important reste la capacité à apprendre de nouveaux langages pour ne pas ``devenir obsolète''.\\

Le travail peut se faire seul ou au sien du équipe, en tant que leader ou membre. L'autonomie est primordiale, la programmation de fait seul. Mais il est très important de savoir travailler en équipe. Par exemple, lorsque nous créons un logiciel dans une équipe, les tâches peuvent être réparties selon les compétences de chacun. Pour assurer une bonne intégration de chaque fonctionnalité, il est nécessaire que les différents membres de l'équipe de développement travaillent entre eux. Concrètement, pour bien faire le lien entre actions de l'utilisateur et affichage, il faut une concertation entre les équipes développant l'affichage et celle celles développant les interactions avec l'utilisateur.\\

La rythme de travail est classique sur des horaires ouvrés, mais avec une certaine autonomie pour leur évolution. En pratique si le développeur préfère travailler entre 15h et 22h, c'est possible. Tant que le travail est fait et que ce dernier continue de participer à la vie du projet (réunions, relecture du code des autres membres de l'équipe, missions diverses, ...).

Les formations exigées ne sont pas arrêtées, l'informatique est un domaine où l'emploi est aisé. Les attentes des recruteurs ne sont pas élevées d'un point de vue diplômes mais plutôt sur des capacités d'apprentissage des technologies spécifiques, des skills de travail en équipe, de l'expérience pratique en entreprise, des projets réels. Pour certaines fonctions, une formation avancée est nécessaire, en recherche et développement par exemple une grande expérience dans le domaine est attendu et l'expérience s'acquiert principalement en stage pendant un cursus universitaire, en école ou institut. Les niveaux de rémunération sont généralement plus élevés que des métiers d'un niveau équivalent et d'une autre branche. Bien sur cela dépend et est principalement vrai avec des évolutions de carrière vers des postes de réflexion et de planification.\\

Dans ce métier, j'affectionne principalement la nécessité de réflexion en amont, pendant et après. Bien que cela puisse être frustrant lorsque un bogue est tenace. Je suis aussi intéressé par les possibilités d'évolution vers des postes de planification et de réflexion. Comme contraintes principales, j'identifie l'aspect de ``diminution physique'' qui est induit par la position assise toute la journée, le travail sur un écran. Comme avantage, je trouve que l'autonomie est un point important, la liberté est aussi importante. Des objectifs sont identifiés, au développeur d'identifier le chemin pour y arriver. Ça me plaît.

\newpage \section{Discussion}
Les informations obtenues durant la phase de recherche documentaire et durant les interview concordent et se complète.\\

La partie salaire, soft et hard skills sont identiques, quoique que plus réels lorsque ces informations sont récupérées directement de la bouche de quelqu'un du métier. 

Sur la partie santé, des informations intéressantes mon été apportées par l'interview de M. COATSALIOU. Elles sont principalement dirigé vers les champs d'applications en recherche chirurgicale, donc pas pertinentes à retranscrire dans ce rapport. 

Grâce à l'interview de M. HERICOURT, j'ai une idée plus précise du quotidien de l'informaticien. De même que les technologies utilisées au CNRS : C, C++, JavaScript, SQL. Et des problèmes rencontrés : création d'interfaces pour l'utilisation de matériel spécifique de recherche, d'outils pour faciliter la communication de ces résultats, ... \\

Les informations que j'ai recueilli sont concordantes avec celles que j'avais auparavant. Elles sont aujourd'hui confortées par des données scientifiques précises (études), par de la documentation officielle et par des retours d'expériences qui rendent le tout beaucoup plus concret. 

J'ai approfondi ma compréhension de cet écosystème, tout en ajoutant des volets que je n'avais pas imaginé, dans la formation ou la recherche et développement.

\newpage \section{Conclusion}
Avec ce travail, j'ai précisé les attendus pour exercer un métier d'ingénieur de formation dans le domaine de la santé. Je regrette de ne pas avoir réussi à identifier un professionnel de ce domaine spécifiquement, d'où mon passage par des chemins détournés pour mêler les deux : informatique et médecine.\\

Maintenant que j'ai une idée plus claire du monde dans lequel m'amène le chemin sur lequel je me suis engagé, de nouvelles interrogations apparaissent concernant des formations annexes ou une expérience supplémentaire. Il me semble que pour intégrer un domaine mêlant étroitement informatique et santé, il pourrait être intéressant de me former aux deux. Actuellement, je suis sur une voie m'amenant à une maîtrise en informatique (si je suis pris en master), mais je suis obligé de délaisser un peu le côté santé au profit du numérique. Pour cela, j'imagine la possibilité de m'engager ensuite ou pendant une césure dans une formation de santé, peut être les neurosciences, qui proposent  un master très ouvert à Bordeaux. je vais devoir approfondir ce sujet. 

En plus de cette formation qui semble intéressante, j'ai remarqué au fil de mes recherches la nécessité d'avoir une première expérience suffisamment forte pour ensuite être engagé dans une entreprise de premier plan. Cela me fait réfléchir étant donné le peu de stage imposé en licence informatique, il pourrait être intéressant d'en réaliser un ou deux ``en plus''.

Pour avoir une idée plus précise des prérequis spécifiques à l'informatique dans le domaine de la santé, je conceptualise la possibilité de me rapprocher de professionnels du milieu, probablement un équipe de recherches de l'IHU Lyric sur la métropole Bordelaise ou une équipe de développeur d'application spécifique santé dans une entreprise (type strava, google fit, apple health, ...).\\

Bien sûr, et comme cela fait parti des soft skills essentiels, je vais continuer à m'auto former aux outils qui ne sont pas pour l'instant enseignés à l'université et à faire une veille informationnelles sur les slack de technologies utilisées par les principaux acteurs du milieu.

\newpage \section{Annexes}
\subsection{Interview 1 de M. PIERRE HERICOURT et M. QUENTIN COATSALIOU}
Liste des questions-réponses du document ``Bilan des interviews''.
\begin{center}
\begin{figure}[h!]
	\includegraphics[scale=1]{interview-total}
\end{figure}
\end{center}




\end{document}